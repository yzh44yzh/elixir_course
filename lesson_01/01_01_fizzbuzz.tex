\chapter{Решаем задачу FizzBuzz}

Начнем знакомство с Эликсир на примере решения задачи  FizzBuzz  https://ru.wikipedia.org/wiki/Fizz\_buzz

Напишите программу, которая выводит на экран числа от 1 до 100. При этом вместо чисел, кратных трем, программа должна выводить слово «Fizz», а вместо чисел, кратных пяти — слово «Buzz». Если число кратно и 3, и 5, то программа должна выводить слово «FizzBuzz».

Это простая задача позволит познакомиться со многими важными элементами языка:
\begin{itemize}
\item модули и функции;
\item генерация списка с помощью `Range`;
\item итерация по списку с помощью `Enum.each`;
\item условые переходы с помощью `cond do`;
\item охранные выражения (guards);
\item вывод на консоль;
\item оператор pipe;
\item и юнит-тесты.
\end{itemize}

\section{Шаг 1. Простая реализация задачи.}

Шаг 1. Простая реализация задачи.

\section{Шаг 2. Отделяем вывод на консоль от логики.}

Шаг 2. Отделяем вывод на консоль от логики.
