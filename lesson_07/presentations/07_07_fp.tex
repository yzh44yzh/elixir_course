\PassOptionsToPackage{x11names}{xcolor}
\documentclass[10pt]{beamer}

\usepackage{fontspec}
\setmainfont{Ubuntu}[]
\setsansfont{Ubuntu}[]
\setmonofont{Ubuntu Mono}[]

\usepackage{graphicx}
\graphicspath{ {../img/} }

\usepackage[absolute,overlay]{textpos} % [showboxes]

\usepackage{listings}
\lstdefinelanguage{elixir}{
    morekeywords={case,catch,def,do,else,false,%
        use,alias,receive,timeout,defmacro,defp,%
        for,if,import,defmodule,defprotocol,%
        nil,defmacrop,defoverridable,defimpl,%
        super,fn,raise,true,try,end,with,%
        unless},
    otherkeywords={<-,->, |>, \%\{, \}, \{, \, (, )},
    sensitive=true,
    morecomment=[l]{\#},
    morecomment=[n]{/*}{*/},
	basicstyle=\ttfamily,
	breaklines,
	showstringspaces=false,
	frame=trbl
}

%% https://latex-tutorial.com/color-latex/
\lstset{
  language=elixir,
  keywordstyle=\color{SteelBlue4},
  identifierstyle=\color{DeepSkyBlue3},
  backgroundcolor=\color{Ivory1}
}

\beamertemplatenavigationsymbolsempty

\title{Что такое функциональное программирование?}

\begin{document}

\begin{frame}
  \frametitle{Что такое функциональное программирование?}
  Не так просто ответить на этот вопрос :)
  \par \bigskip
  Есть языки программирования,
  \par \bigskip
  которые традиционно относят к ФП:
  \par \bigskip
  \begin{itemize}
  \item Lisp
  \item Haskell
  \item Standard ML
  \item OCaml
  \item Scala
  \end{itemize}
\end{frame}

\begin{frame}
  \frametitle{Что такое функциональное программирование?}
  Однако многие из них мультипарадигменные.
  \par \bigskip
  То есть, они позволяют писать код не только в стиле ФП.
\end{frame}

\begin{frame}
  \frametitle{Что такое функциональное программирование?}
  С другой стороны есть языки,
  \par \bigskip
  которые изначально не поддерживали ФП,
  \par \bigskip
  но со временем позаимствовали оттуда много элементов:
  \par \bigskip
  \begin{itemize}
  \item Java
  \item C++
  \item Python
  \item JavaScript
  \end{itemize}
\end{frame}

\begin{frame}
  \frametitle{Что такое функциональное программирование?}
  Парадигмы смешиваются в рамках одного языка.
  \par \bigskip
  Но мы всё-таки можем отличить,
  \par \bigskip
  где код написан в стиле ФП, а где нет.
\end{frame}

\begin{frame}
  \frametitle{Элементы ФП}
  Для функционального программирования характерно:
  \par \bigskip
  \begin{itemize}
  \item Иммутабельные данные (Immutability);
  \item Рекурсия, как основной способ итерации по коллекциям;
  \item Функции высшего порядка (HOF);
  \item Анонимные функции (замыкания, лямбды);
  \item Алгебраические типы данных (ADT);
  \item Сопоставление с образцом (Pattern Matching);
  \item Ленивые вычисления (Lazy Evaluation).
  \end{itemize}
  \par \bigskip
  Всё это есть в Эликсир.
\end{frame}

\begin{frame}
  \frametitle{Элементы ФП}
  А вот этого нет в Эликсир:
  \par \bigskip
  \begin{itemize}
  \item Автоматический вывод типов (Type Inference);
  \item Чистые функции и контроль побочных эффектов.
  \end{itemize}
\end{frame}

\begin{frame}
  \frametitle{Иммутабельность}
  Исключает класс ошибок,
  \par \bigskip
  связанных с модификацией одной области памяти
  \par \bigskip
  из разных мест в коде.
  \par \bigskip
  Упрощает многие аспекты разработки:
  \par \bigskip
  \begin{itemize}
  \item многопоточность,
  \item отладку,
  \item статический анализ кода.
  \end{itemize}
\end{frame}

\begin{frame}
  \frametitle{Рекурсия}
  Итерация на основе рекурсии.
  \par \bigskip
  Функции высшего порядка: map, filter, reduce.
  \par \bigskip
  В этих элементах ФП языки наиболее похожи друг на друга.
  \par \bigskip
  Их синтаксис может сильно отличаться,
  \par \bigskip
  но семантика остается одна и та же.
\end{frame}

\begin{frame}
  \frametitle{Анонимные функции}
  Анонимные функции и замыкания дополняют HOF
  \par \bigskip
  и добавляют удобства в разработке.
  \par \bigskip
  Эти элементы часто встречаются и в не-ФП языках,
  \par \bigskip
  но там они могут выполнять другую роль.
\end{frame}

\begin{frame}
  \frametitle{Сопоставление с образцом}
  Сопоставление с образцом (Pattern Matching) --
  \par \bigskip
  основной способ реализации условных переходов.
  \par \bigskip
  Сильно отличает ФП код от императивного.
\end{frame}

\begin{frame}
  \frametitle{ADT}
  Алгебраические типы данных определяют то,
  \par \bigskip
  как мы моделируем сущности в нашей программе.
  \par \bigskip
  Четкое разделение данных и функций отличает ФП от ООП.
\end{frame}

\begin{frame}
  \frametitle{Ленивые вычисления}
  Ассоциируются в первую очередь с языком Хаскель.
  \par \bigskip
  В Хаскель любые вычисения являются ленивыми,
  \par \bigskip
  что отличает его от большинства других языков,
  \par \bigskip
  в т.ч. функциональных.
\end{frame}

\begin{frame}
  \frametitle{Автоматический вывод типов}
  Сейчас есть во многих мейнстримовых языках,
  \par \bigskip
  например, в C++ и Java.
  \par \bigskip
  В 80-90-е годы это было характерно для ФП,
  \par \bigskip
  а в мейнстрим проникло позже.
\end{frame}

\begin{frame}
  \frametitle{Контроль побочных эффектов}
  В большинстве языков чистота функций,
  \par \bigskip
  отсутствие побочных эффектов --
  \par \bigskip
  это ответсвенность разработчика.
  \par \bigskip
  В Хаскель это контролирует компилятор.
\end{frame}

\begin{frame}
  \frametitle{Чистый код}
  \begin{itemize}
  \item легче понять
  \item легче компоновать
  \item содержит меньше ошибок
  \end{itemize}
\end{frame}

\begin{frame}
  \frametitle{Чистый код}
  В Эликсир чистый код легко писать на уровне одного потока.
  \par \bigskip
  Но взаимодействие между потоками это уже побочный эффект.
\end{frame}

\begin{frame}
  \frametitle{Что такое ФП?}
  Код, который \textbf{в основном} состоит из элементов ФП.
  \par \bigskip
  Хотя в нём могут быть не-ФП элементы:
  \begin{itemize}
  \item мутабельная память,
  \item циклы for вместо рекурсии,
  \item побочные эффекты.
  \end{itemize}
  \par \bigskip
  Дело в пропорции тех и других элементов.
\end{frame}

\end{document}
