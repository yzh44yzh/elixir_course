\documentclass[10pt]{beamer}

\usepackage{fontspec}
\setmainfont{Ubuntu}[]
\setsansfont{Ubuntu}[]
\setmonofont{Ubuntu Mono}[]

\usepackage{graphicx}
\graphicspath{ {../img/} }

\usepackage[absolute,overlay]{textpos} % [showboxes]

\beamertemplatenavigationsymbolsempty

\title{Стандартные протоколы}

\begin{document}

\begin{frame}
  \frametitle{Стандартные протоколы}
  В Elixir есть 5 стандартных протоколов:
  \par \bigskip
  \begin{itemize}
  \item Enumerable
  \item Collectable
  \item Inspect
  \item List.Chars
  \item String.Chars
  \end{itemize}
\end{frame}

\begin{frame}
  \frametitle{Enumerable}
  Все коллекции реализуют протокол Enumerable.
  \par \bigskip
  Поэтому функции модулей \textbf{Enum} и \textbf{Stream} работают с:
  \par \bigskip
  \begin{itemize}
  \item List
  \item Keyword List
  \item Map
  \item MapSet
  \item Range
  \item String
  \end{itemize}
\end{frame}

\begin{frame}
  \frametitle{Enumerable}
  Протокол содержит 4 функции:
  \par \bigskip
  \begin{itemize}
  \item count/1
  \item member?/2
  \item reduce/3
  \item slice/1
  \end{itemize}
\end{frame}

\begin{frame}
  \frametitle{Enumerable}
  Напомню, что функций в модуле \textbf{Enum} много:
  \par \bigskip
  \begin{itemize}
  \item map, filter, reduce
  \item all?, any?
  \item sort, find
  \item drop\_while, dedup\_by
  \item chunk\_by, chunk\_while
  \item и другие
  \end{itemize}
  \par \bigskip
  Все они реализуются через 4 функции протокола.
\end{frame}

\begin{frame}
  \frametitle{Enumerable}
  Теоретически достаточно только \textbf{reduce}.
  \par \bigskip
  Все остальное можно реализовать через него.
  \par \bigskip
  Но на практике это не эффективно.
\end{frame}

\begin{frame}
  \frametitle{Enumerable}
  \textbf{Enumerable.reduce} это не то же самое, что \textbf{Enum.reduce}.
  \par \bigskip
  Там более сложная реализация, где можно управлять итерацией:
  останавливать, возобновлять и прерывать.
\end{frame}

\end{document}
