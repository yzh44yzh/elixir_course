\chapter{Решаем задачу FizzBuzz}

Начнем знакомство с Эликсир на примере решения задачи FizzBuzz:

\textit{Напишите программу, которая выводит на экран числа от 1 до 100. При этом вместо чисел, кратных трем, программа должна выводить слово «Fizz», а вместо чисел, кратных пяти — слово «Buzz». Если число кратно и 3, и 5, то программа должна выводить слово «FizzBuzz».}

Это простая задача позволит познакомиться со многими важными элементами языка:
\begin{itemize}
\item модули и функции;
\item генерация списка с помощью \texttt{Range};
\item итерация по списку с помощью \texttt{Enum.each};
\item условые переходы с помощью \texttt{cond do};
\item охранные выражения (guards);
\item вывод на консоль;
\item оператор pipe;
\item и модульные тесты (unit tests).
\end{itemize}

\section{Шаг 1. Простая реализация задачи.}

Создаем модуль \textbf{FizzBuzz01} и в нем две функции \texttt{main} и \texttt{fizzbuzz}.

TODO: listing here

В функции \texttt{main} мы генерируем последовательность от 1 до 100. Конструкция \texttt{1..100} -- это генератор последовательности, он называется \textbf{Range}. Затем с помощью \texttt{Enum.each} мы применяем функцию \texttt{fizzbuzz} к каждому элементу.

\texttt{fizzbuzz} использует конструкцию \texttt{cond do} и охранные выражения (guards) чтобы проверить условия делимости на 3 и на 5. Первое охранное выражение выполняется, если \texttt{n} делится на 3 и на 5. Второе охранное выражение выполняется, если \texttt{n} делится на 3. Третье, если \texttt{n} делится на 5. И последнее, четвертое охранное выражение выполняется всегда, так как оно представлено просто значением \texttt{true}.

Функция \texttt{rem}, как не трудно догадаться, возвращает остаток от деления.

\texttt{cond do} проверяет выражения по очереди, и выполняет только одну ветку кода, соответствующую первому истинному выражению. \texttt{IO.puts} выводит нужное значение на стандартный вывод.
