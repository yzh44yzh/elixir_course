\documentclass[10pt]{beamer}

\usepackage{fontspec}
\setmainfont{Ubuntu}[]
\setsansfont{Ubuntu}[]
\setmonofont{Ubuntu Mono}[]

\usepackage{graphicx}
\graphicspath{ {../img/} }

\usepackage{listings}

%% \usepackage[absolute,overlay]{textpos}[showboxes]

\beamertemplatenavigationsymbolsempty

\title{Устройство списков}

\begin{document}

\begin{frame}[fragile]
  \frametitle{Создание списка}
  \begin{lstlisting}
  my_list = [1, 2, 3, 4]
  \end{lstlisting}
\end{frame}

\begin{frame}
  \frametitle{Создание списка}
  \centering
  \includegraphics[scale=0.6]{list_1}
\end{frame}

\begin{frame}
  \frametitle{Добавление нового элемента}
  \centering
  \includegraphics[scale=0.6]{list_2}
\end{frame}

\begin{frame}[fragile]
  \frametitle{Список как последовательность операторов cons}
  \begin{lstlisting}
  [1 | [2 | [3 | [4 | []]]]]
  \end{lstlisting}
\end{frame}

\begin{frame}
  \frametitle{Список как последовательность операторов cons}
  \centering
  \includegraphics[scale=0.6]{list_3}
\end{frame}

\begin{frame}
  \frametitle{Создание двунаправленного списка}
  \centering
  \includegraphics[scale=0.6]{list_4}
\end{frame}

\begin{frame}
  \frametitle{Добавление нового элемента}
  \centering
  \includegraphics[scale=0.6]{list_5}
\end{frame}

\begin{frame}
  \frametitle{Полное копирование списка}
  \centering
  \includegraphics[scale=0.6]{list_6}
\end{frame}

\begin{frame}
  \frametitle{Двунаправленный список}
  В нём есть циклические ссылки.
  \par \bigskip
  Из-за них невозможно переиспользование памяти
  \par
  (structure sharing).
  \par \bigskip
  И при любом изменении данных нужно полностью их копировать в другое место в памяти.
\end{frame}

\begin{frame}
  \frametitle{Стоимость операций для списков}
  \begin{itemize}
  \item добавить элемент в начало списка - O(1)
  \item добавить элемент в конец списка - O(n)
  \item определить длину списка - O(n)
  \item получить N-й элемент - O(n)
  \end{itemize}
\end{frame}

\begin{frame}[fragile]
  \frametitle{Стоимость сложения двух списков}
  \begin{lstlisting}
    [1, 2, 3] ++ [4, 5, 6]
  \end{lstlisting}
  \par \bigskip
  O(n), где \textbf{n} -- длина первого списка
\end{frame}

\begin{frame}
  \frametitle{Стоимость операций для массивов}
  \begin{itemize}
  \item добавить элемент в начало массива - O(n)
  \item добавить элемент в конец массива - O(1)/O(n)
  \item определить длину массива - O(1)
  \item получить/модифицировать N-й элемент - O(1)
  \end{itemize}
\end{frame}

\begin{frame}
  \frametitle{Главное преимущество массивов}
  Константное время доступа к любому элементу.
\end{frame}

\end{document}
