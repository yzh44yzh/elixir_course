\PassOptionsToPackage{x11names}{xcolor}
\documentclass[10pt]{beamer}

\usepackage{fontspec}
\setmainfont{Ubuntu}[]
\setsansfont{Ubuntu}[]
\setmonofont{Ubuntu Mono}[]

\usepackage[absolute,overlay]{textpos} % [showboxes]

\usepackage{listings}
\lstdefinelanguage{elixir}{
    morekeywords={case,catch,def,do,else,false,%
        use,alias,receive,timeout,defmacro,defp,%
        for,if,import,defmodule,defprotocol,%
        nil,defmacrop,defoverridable,defimpl,%
        super,fn,raise,true,try,end,with,%
        unless},
    otherkeywords={<-,->, |>, \%\{, \}, \{, \, (, )},
    sensitive=true,
    morecomment=[l]{\#},
    morecomment=[n]{/*}{*/},
	basicstyle=\ttfamily,
	breaklines,
	showstringspaces=false,
	frame=trbl
}

%% https://latex-tutorial.com/color-latex/
\lstset{
  language=elixir,
  keywordstyle=\color{SteelBlue4},
  identifierstyle=\color{DeepSkyBlue3},
  backgroundcolor=\color{Ivory1}
}

\beamertemplatenavigationsymbolsempty

\title{Что такое монада?}

\begin{document}

\begin{frame}
  \frametitle{Что такое монада?}
  Копнём функциональное программирование глубже:
  \par \bigskip
  \begin{itemize}
  \item Монады,
  \item Каррирование,
  \item Монадные трансформеры.
  \end{itemize}
\end{frame}

\begin{frame}
  \frametitle{Элементы ФП}
  Из 7-го урока:
  \par \bigskip
  \begin{itemize}
  \item Иммутабельные данные (Immutability);
  \item Рекурсия, как основной способ итерации по коллекциям;
  \item Функции высшего порядка (HOF);
  \item Анонимные функции (замыкания, лямбды);
  \item Алгебраические типы данных (ADT);
  \item Сопоставление с образцом (Pattern Matching);
  \item Ленивые вычисления (Lazy Evaluation).
  \item Автоматический вывод типов (Type Inference);
  \item Чистые функции и контроль побочных эффектов.
  \end{itemize}
\end{frame}

\begin{frame}
  \frametitle{Что такое ФП?}
  Код, который \textbf{в основном} состоит из элементов ФП.
  \par \bigskip
  Хотя в нём могут быть не-ФП элементы:
  \begin{itemize}
  \item мутабельная память,
  \item циклы for вместо рекурсии,
  \item побочные эффекты.
  \end{itemize}
  \par \bigskip
  Дело в пропорции тех и других элементов.
\end{frame}

\begin{frame}[fragile]
  \frametitle{Что такое монада?}
  Монада -- это значение, обернутое в некоторый контекст.
  \par \bigskip
  \begin{lstlisting}
{:ok, book} =
  Bookshop.Controller.validate_book(data)
  \end{lstlisting}
  Значение \textbf{book} обернуто в контекст \textbf{\{:ok, ...\}}.
  \par \bigskip
  \begin{lstlisting}
{:error, :book_not_found} =
  Bookshop.Controller.validate_book(data)
  \end{lstlisting}
  Значение \textbf{:book\_not\_found} обернуто в контекст \textbf{\{:error, ...\}}.
\end{frame}

\begin{frame}[fragile]
  \frametitle{Монада Result}
  Такой тип данных:
  \par \bigskip
  \begin{lstlisting}
{:ok, successful_value} | {:error, error_value}
  \end{lstlisting}
  \par \bigskip
  Это монада \textbf{Result} -- одна из самых популярных монад.
\end{frame}

\begin{frame}[fragile]
  \frametitle{Монада Result}
  Есть практически во всех ФП языках и во многих не ФП языках:
  \begin{itemize}
  \item Haskell \textbf{Result e t}
  \item OCaml \textbf{('e, 't) result}
  \item Rust \textbf{Result<T, E>}
  \item Scala \textbf{Either[E, T]}
  \end{itemize}
\end{frame}

\begin{frame}[fragile]
  \frametitle{Монада Maybe}
  Тоже встречается часто:
  \begin{lstlisting}
> m = %{a: 42}
> Map.fetch(m, :a)
{:ok, 42}
> Map.fetch(m, :b)
:error
  \end{lstlisting}
  \par \bigskip
  Описывается типом:
  \begin{lstlisting}
{:ok, value} | :no_value
  \end{lstlisting}
\end{frame}

\begin{frame}
  \frametitle{Монада Maybe}
  Это упрощённый Result. 
  \par \bigskip
  Контекст здесь в том, что значение может быть,
  \par \bigskip
  или может отсутстовать.
\end{frame}

\begin{frame}[fragile]
  \frametitle{Монада Future}
  \begin{lstlisting}
> future_value = Task.async(fn() -> 42 end)
%Task{
  owner: #PID<0.107.0>,
  pid: #PID<0.113.0>,
  ref: #Reference<0.3496306138.1615593474.196666>
}
> Task.await(future_value)
42
  \end{lstlisting}
\end{frame}

\begin{frame}
  \frametitle{Монада Future}
  Это асинхронное вычисление.
  \par \bigskip
  Значение изначально отпутствует.
  \par \bigskip
  Контекст позволяет получить его позже, когда оно появится.
\end{frame}

\begin{frame}[fragile]
  \frametitle{Оператор Bind}
  В 4-м решеним мы реализовали функцию bind:
  \par \bigskip
  \begin{lstlisting}
@spec bind(m_fun(), m_fun()) :: m_fun()
def bind(f1, f2) do
  fn args ->
    case f1.(args) do
      {:ok, result} -> f2.(result)
      {:error, error} -> {:error, error}
    end
  end
end
  \end{lstlisting}
\end{frame}

\begin{frame}[fragile]
  \frametitle{Оператор Bind}
  И применяли её для связывания функций:
  \par \bigskip
  \begin{lstlisting}
f =
  FP.bind(&C.validate_incoming_data/1,
          &step_validate_cat/1)
  |> FP.bind(&step_validate_address/1)
  |> FP.bind(&step_validate_books/1)
  |> FP.bind(&step_create_order/1)

f.(data)
  \end{lstlisting}
\end{frame}

\begin{frame}
  \frametitle{Оператор Bind}
  Наша функция работает только с монадой \textbf{Result}.
  \par \bigskip
  Не сложно сделать похожую функцию для \textbf{Maybe}.
  \par \bigskip
  А вот сделать связывание для \textbf{Future} уже сложнее.
\end{frame}

\begin{frame}[fragile]
  \frametitle{Оператор Bind}
  Допустим, у нас есть несколько асинхронных запросов:
  \par \bigskip
  \begin{lstlisting}
future1 = request_service_1()
value1 = Task.await(future1)

future2 = request_service_2()
value2 = Task.await(future2)

future3 = request_service_3()
value3 = Task.await(future3)
  \end{lstlisting}
\end{frame}

\begin{frame}[fragile]
  \frametitle{Оператор Bind}
  И мы хотим реализовать такой bind,
  \par \bigskip
  который свяжет эти запросы:
  \par \bigskip
  \begin{lstlisting}
bind_async(
  &request_service_1/0, 
  &request_service_2/0
)
|> bind_async(&request_service_3/0)
  \end{lstlisting}
\end{frame}

\begin{frame}
  \frametitle{Оператор Bind}
  Простая реализация, с блокировкой после каждого вызова.
  \par \bigskip
  Сложная реализация, с паралельными вызовами.
\end{frame}

\begin{frame}
  \frametitle{Что общего у всех монад?}
  Почему мы собрали \textbf{Result} и \textbf{Future} в одну группу?
  \par \bigskip
  Что у них общего?
\end{frame}

\begin{frame}[fragile]
  \frametitle{Что общего у всех монад?}
  Общее то, что для них можно сделать композицию функций
  \par \bigskip
  с помощью \textbf{bind}:
  \par \bigskip
  \begin{lstlisting}
  validate_book
  >>= buy_book
  >>= deliver_book

  request_service_1
  >>= request_service_2
  >>= request_service_3
  \end{lstlisting}
\end{frame}

\begin{frame}
  \frametitle{Что общего у всех монад?}
  При желании в Эликсире тоже можно реализовать такой bind,
  \par \bigskip
  который будет работать с любыми монадами.
  \par \bigskip
  Есть и библиотека, которая делает нечто подобное:
  \par \bigskip
  \textcolor{DeepSkyBlue3}{https://hexdocs.pm/monad/Monad.html}
\end{frame}

\begin{frame}
  \frametitle{Другие операции над монадами}
  Кроме bind возможны и другие операции над монадами.
  \par \bigskip
  Обозначим простое значение как \textbf{v},
  \par \bigskip
  а значение внутри монады как \textbf{M<v>}.
\end{frame}

\begin{frame}[fragile]
  \frametitle{Другие операции над монадами}
  Для 2-х вариантов значений могут быть 4 варианта функций:
  \par \bigskip
  \begin{lstlisting}
f1(v) :: v
f2(v) :: M<v>
f3(M<v>) :: v
f4(M<v>) :: M<v>
  \end{lstlisting}
\end{frame}

\begin{frame}[fragile]
  \frametitle{Другие операции над монадами}
  Все эти варианты нужно комбинировать между собой:
  \par \bigskip
  \begin{lstlisting}
  f1 |> f1
  f2 >>= f2
  f3 |> wrap_into_monad |> f3
  f4 |> f4
  \end{lstlisting}
  \par \bigskip
  В Хаскеле 1й и 4й варианты отличаются, а в Эликсир нет.
  \par \bigskip
  \begin{lstlisting}
  f1 . f1
  f2 >>= f2
  f3 . return . f3
  f4 >> f4
  \end{lstlisting}
\end{frame}

\begin{frame}[fragile]
  \frametitle{ФП -- это композиция функций}
В теории любая программа сводится к цепочке функций,
  \par \bigskip
скомпонованых друг с другом разными способами:
  \par \bigskip
  \begin{lstlisting}
  f1 >>= f2 . f3 >> f4 >>= f5 . f6
  \end{lstlisting}
\end{frame}

\begin{frame}
  \frametitle{ФП -- это композиция функций}
Цепочка представляет собой happy path,
  \par \bigskip
а операторы композиции скрывают ветвления.
\end{frame}

\begin{frame}
  \frametitle{Каррирование}
  Пока мы видели функции с одним аргументом.
  \par \bigskip
  А что, если аргументов больше?
\end{frame}

\begin{frame}
  \frametitle{Каррирование}
  Здесь помогает каррирование.
  \par \bigskip
  Функцию f/3 можно представить как функцию f/1,
  \par
  возвращающую f/2.
  \par \bigskip
  А функцию f/2 можно представить, как f/1,
  \par
  возвращающую f/1.
\end{frame}

\begin{frame}[fragile]
  \frametitle{Каррирование}
  \begin{lstlisting}
def f2(a) do
  fn(b) -> a + b end
end

def f3(a) do
  fn(b) ->
    fn(c) -> a + b + c end
  end
end
  \end{lstlisting}
\end{frame}

\begin{frame}[fragile]
  \frametitle{Каррирование}
  \begin{lstlisting}
> f1 = f2(1)
> res = f1.(2)

> res = f2(1).(2)

> f2 = f3(1)
> f1 = f2.(2)
> res = f1.(3)

> res = f3(1).(2).(3)
  \end{lstlisting}
\end{frame}

\begin{frame}
  \frametitle{Каррирование}
  Любую функцию можно представить как функцию
  \par
  от одного аргумента.
  \par \bigskip
  В Эликсир это возможно, но совершенно непрактично.
  \par \bigskip
  В Хаскель это работает автоматически для любой функции.
  \par \bigskip
  За счёт каррирования мы можем применять
  \par
  все способы композиции ко всем функциям.
\end{frame}

\begin{frame}
  \frametitle{}
\end{frame}

\begin{frame}
  \frametitle{}
\end{frame}

\end{document}
