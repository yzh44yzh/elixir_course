\PassOptionsToPackage{x11names}{xcolor}
\documentclass[10pt]{beamer}

\usepackage{fontspec}
\setmainfont{Ubuntu}[]
\setsansfont{Ubuntu}[]
\setmonofont{Ubuntu Mono}[]

\usepackage[absolute,overlay]{textpos} % [showboxes]

\usepackage{listings}
\lstdefinelanguage{elixir}{
    morekeywords={case,catch,def,do,else,false,%
        use,alias,receive,timeout,defmacro,defp,%
        for,if,import,defmodule,defprotocol,%
        nil,defmacrop,defoverridable,defimpl,%
        super,fn,raise,true,try,end,with,%
        unless},
    otherkeywords={<-,->, |>, \%\{, \}, \{, \, (, )},
    sensitive=true,
    morecomment=[l]{\#},
    morecomment=[n]{/*}{*/},
	basicstyle=\ttfamily,
	breaklines,
	showstringspaces=false,
	frame=trbl
}

%% https://latex-tutorial.com/color-latex/
\lstset{
  language=elixir,
  keywordstyle=\color{SteelBlue4},
  identifierstyle=\color{DeepSkyBlue3},
  backgroundcolor=\color{Ivory1}
}

\beamertemplatenavigationsymbolsempty

\title{Что такое монада?}

\begin{document}

\begin{frame}
  \frametitle{Что такое монада?}
  Копнём функциональное программирование глубже:
  \par \bigskip
  \begin{itemize}
  \item Монады,
  \item Каррирование,
  \item Монадные трансформеры.
  \end{itemize}
\end{frame}

\begin{frame}
  \frametitle{Элементы ФП}
  Из 7-го урока:
  \par \bigskip
  \begin{itemize}
  \item Иммутабельные данные (Immutability);
  \item Рекурсия, как основной способ итерации по коллекциям;
  \item Функции высшего порядка (HOF);
  \item Анонимные функции (замыкания, лямбды);
  \item Алгебраические типы данных (ADT);
  \item Сопоставление с образцом (Pattern Matching);
  \item Ленивые вычисления (Lazy Evaluation).
  \item Автоматический вывод типов (Type Inference);
  \item Чистые функции и контроль побочных эффектов.
  \end{itemize}
\end{frame}

\begin{frame}
  \frametitle{Что такое ФП?}
  Код, который \textbf{в основном} состоит из элементов ФП.
  \par \bigskip
  Хотя в нём могут быть не-ФП элементы:
  \begin{itemize}
  \item мутабельная память,
  \item циклы for вместо рекурсии,
  \item побочные эффекты.
  \end{itemize}
  \par \bigskip
  Дело в пропорции тех и других элементов.
\end{frame}

\begin{frame}[fragile]
  \frametitle{Что такое монада?}
  Монада -- это значение, обернутое в некоторый контекст.
  \par \bigskip
  \begin{lstlisting}
{:ok, book} =
  Bookshop.Controller.validate_book(data)
  \end{lstlisting}
  Значение \textbf{book} обернуто в контекст \textbf{\{:ok, ...\}}.
  \par \bigskip
  \begin{lstlisting}
{:error, :book_not_found} =
  Bookshop.Controller.validate_book(data)
  \end{lstlisting}
  Значение \textbf{:book\_not\_found} обернуто в контекст \textbf{\{:error, ...\}}.
\end{frame}

\begin{frame}[fragile]
  \frametitle{Монада Result}
  Такой тип данных:
  \par \bigskip
  \begin{lstlisting}
{:ok, successful_value} | {:error, error_value}
  \end{lstlisting}
  \par \bigskip
  Это монада \textbf{Result} -- одна из самых популярных монад.
\end{frame}

\begin{frame}[fragile]
  \frametitle{Монада Result}
  Есть практически во всех ФП языках и во многих не ФП языках:
  \begin{itemize}
  \item Haskell \textbf{Result e t}
  \item OCaml \textbf{('e, 't) result}
  \item Rust \textbf{Result<T, E>}
  \item Scala \textbf{Either[E, T]}
  \end{itemize}
\end{frame}

\begin{frame}[fragile]
  \frametitle{Монада Maybe}
  Тоже встречается часто:
  \begin{lstlisting}
> m = %{a: 42}
> Map.fetch(m, :a)
{:ok, 42}
> Map.fetch(m, :b)
:error
  \end{lstlisting}
  \par \bigskip
  Описывается типом:
  \begin{lstlisting}
{:ok, value} | :no_value
  \end{lstlisting}
\end{frame}

\begin{frame}
  \frametitle{Монада Maybe}
  Это упрощённый Result. 
  \par \bigskip
  Контекст здесь в том, что значение может быть,
  \par \bigskip
  или может отсутстовать.
\end{frame}

\begin{frame}[fragile]
  \frametitle{Монада Future}
  \begin{lstlisting}
> future_value = Task.async(fn() -> 42 end)
%Task{
  owner: #PID<0.107.0>,
  pid: #PID<0.113.0>,
  ref: #Reference<0.3496306138.1615593474.196666>
}
> Task.await(future_value)
42
  \end{lstlisting}
\end{frame}

\begin{frame}
  \frametitle{Монада Future}
  Это асинхронное вычисление.
  \par \bigskip
  Значение изначально отпутствует.
  \par \bigskip
  Контекст позволяет получить его позже, когда оно появится.
\end{frame}

\begin{frame}[fragile]
  \frametitle{Оператор Bind}
  В 4-м решеним мы реализовали функцию bind:
  \par \bigskip
  \begin{lstlisting}
@spec bind(m_fun(), m_fun()) :: m_fun()
def bind(f1, f2) do
  fn args ->
    case f1.(args) do
      {:ok, result} -> f2.(result)
      {:error, error} -> {:error, error}
    end
  end
end
  \end{lstlisting}
\end{frame}

\begin{frame}[fragile]
  \frametitle{Оператор Bind}
  И применяли её для связывания функций:
  \par \bigskip
  \begin{lstlisting}
f =
  FP.bind(&C.validate_incoming_data/1,
          &step_validate_cat/1)
  |> FP.bind(&step_validate_address/1)
  |> FP.bind(&step_validate_books/1)
  |> FP.bind(&step_create_order/1)

f.(data)
  \end{lstlisting}
\end{frame}

\begin{frame}
  \frametitle{Оператор Bind}
  Наша функция работает только с монадой \textbf{Result}.
  \par \bigskip
  Не сложно сделать похожую функцию для \textbf{Maybe}.
  \par \bigskip
  А вот сделать связывание для \textbf{Future} уже сложнее.
\end{frame}

\end{document}
