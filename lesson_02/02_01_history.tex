\chapter{Немного истории}

Эликсир молодой язык, он создан в 2012 году. Но начать я хочу за 100 лет до этого. 

\section{Предыстория: начало ХХ века.}

TODO image
Agner Krarup Erlang % ./img/agner_krarup_erlang.jpg

В начале ХХ века в Дании, в Копенгагенской телефонной компании работал инженер \href{https://ru.wikipedia.org/wiki/%D0%AD%D1%80%D0%BB%D0%B0%D0%BD%D0%B3,_%D0%90%D0%B3%D0%BD%D0%B5%D1%80_%D0%9A%D1%80%D0%B0%D1%80%D1%83%D0%BF}{Агнер Краруп Эрланг}

Его интересовало, как наиболее эффективным образом использовать оборудование телефонной станции, чтобы обслуживать максимальное количество абонентов.

Агнер Эрланг был хорошо образован, с отличием закончил университет, и умел применять в работе методы математики и статистики. И он не ленился самостоятельно залезать на телефонные столбы, чтобы собрать нужные данные.

Результатом его усилий стала научная работа \href{https://ru.wikipedia.org/wiki/%D0%A2%D0%B5%D0%BE%D1%80%D0%B8%D1%8F_%D0%BC%D0%B0%D1%81%D1%81%D0%BE%D0%B2%D0%BE%D0%B3%D0%BE_%D0%BE%D0%B1%D1%81%D0%BB%D1%83%D0%B6%D0%B8%D0%B2%D0%B0%D0%BD%D0%B8%D1%8F}{"Теория массового обслуживания"}. Она позволяет рационально расчитать ресурсы, необходимые для обслуживания требований, поступающих в систему, исходя из длительности ожидания и длины очередей.

Теория применяется не только в телекоммуникационных системах, а гораздо шире: в управлении автомобильным и воздушным движением, на конвейерном производстве, в логистике, а также при проектировании фабрик, складов, магазинов и больниц.

В честь Агнера Эрланга названы единица измерения трафика в телекоммуникационных системах и язык программирования.
