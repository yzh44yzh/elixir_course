\documentclass[10pt]{beamer}

\usepackage{fontspec}
\setmainfont{Ubuntu}[]
\setsansfont{Ubuntu}[]
\setmonofont{Ubuntu Mono}[]

\usepackage{graphicx}
\graphicspath{ {../img/} }

\usepackage{listings}
\lstset{
  language=ML,
  keywordstyle=\color{blue},
  backgroundcolor=\color{lightgray}
}

\beamertemplatenavigationsymbolsempty

\title{Свойства Эликсир}

\begin{document}

\begin{frame}
  \frametitle{Свойства Эликсир}
  \begin{itemize}
  \item Мощная система макросов
  \item Высокоуровневый язык
  \item Экосистема и сообщество
  \end{itemize}
\end{frame}

\begin{frame}
  \frametitle{Система макросов}
  Работает на уровне абстрактного синтаксического дерева.
  \par \bigskip
  Позволяет расширять язык и строить Domain~Specific~Languages.
  \par \bigskip
  Примеры: \textbf{Ecto} и \textbf{Phoenix}.
\end{frame}

\begin{frame}
  \frametitle{Высокоуровневый язык}
  \begin{itemize}
  \item Оператор \textbf{pipe}
  \item Макрос \textbf{with}
  \item Абстракции Эликсир
  \item Абстракции Эрланг
  \end{itemize}  
\end{frame}


\begin{frame}[fragile]
  \frametitle{Оператор pipe}
  pipe
  \begin{lstlisting}
    a = func1()
    b = func2(a)
    c = func3(b)
  \end{lstlisting}
\end{frame}

\begin{frame}
  \frametitle{Макрос with}
  with
\end{frame}

\begin{frame}
  \frametitle{Абстракции Эликсир}
  Enum
\end{frame}

\begin{frame}
  \frametitle{Абстракции Эрланг}
  HOF
\end{frame}

\begin{frame}
  \frametitle{Экосистема и сообщество}
\end{frame}

\end{document}
