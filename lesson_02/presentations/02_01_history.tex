\documentclass[10pt]{beamer}
\usepackage{fontspec}
\usepackage{listings}

\setmainfont{Ubuntu}[]
\setsansfont{Ubuntu}[]
\setmonofont{Ubuntu Mono}[]

\usetheme{Singapore}
\usecolortheme{dove}
\beamertemplatenavigationsymbolsempty
\setbeamertemplate{headline}{}

\lstset{
  language=ML,
  keywordstyle=\color{blue},
  backgroundcolor=\color{lightgray}
}

\title{Немного истории}

\begin{document}
\maketitle

\begin{frame}
\frametitle{Агенда}
\centering
Предыстория: начало ХХ века.
\par \bigskip
История Erlang: c 1985 по настоящее время.
\par \bigskip
История Elixir: c 2011 по настоящее время.
\end{frame}


\section{Предыстория: начало ХХ века.}

{
\setbeamercolor{background canvas}{bg=orange}
\begin{frame}
\frametitle{Чуть-чуть истории}
\centering
Это важно для понимания сути.
\end{frame}
}

\begin{frame}
\frametitle{Агнер Краруп Эрланг}
\centering
Датский математик, статистик и инженер,
\par \bigskip
автор "Теории массового обслуживания".
\end{frame}

\begin{frame}
\frametitle{Теория массового обслуживания}
\centering
1909
\par \bigskip
Теория очередей, Queueing theory
\par \bigskip
Математическая модель для оценки пропускной способности телекоммуникационных сетей
\end{frame}

\end{document}
