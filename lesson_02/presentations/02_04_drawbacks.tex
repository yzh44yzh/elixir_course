\PassOptionsToPackage{x11names}{xcolor}
\documentclass[10pt]{beamer}

\usepackage{fontspec}
\setmainfont{Ubuntu}[]
\setsansfont{Ubuntu}[]
\setmonofont{Ubuntu Mono}[]

\usepackage{graphicx}
\graphicspath{ {../img/} }

\usepackage{listings}
\lstdefinelanguage{elixir}{
    morekeywords={case,catch,def,do,else,false,%
        use,alias,receive,timeout,defmacro,defp,%
        for,if,import,defmodule,defprotocol,%
        nil,defmacrop,defoverridable,defimpl,%
        super,fn,raise,true,try,end,with,%
        unless},
    otherkeywords={<-,->, |>, \%\{, \}, \{, \, (, )},
    sensitive=true,
    morecomment=[l]{\#},
    morecomment=[n]{/*}{*/},
	basicstyle=\ttfamily,
	breaklines,
	showstringspaces=false,
	frame=trbl
}

%% https://latex-tutorial.com/color-latex/
\lstset{
  language=elixir,
  keywordstyle=\color{SteelBlue4},
  identifierstyle=\color{DeepSkyBlue3},
  backgroundcolor=\color{Ivory1}
}

\beamertemplatenavigationsymbolsempty

\title{Недостатки Эликсир}

\begin{document}

\begin{frame}
  \frametitle{Недостатки BEAM и Эликсир}
  \begin{itemize}
  \item Недостатки BEAM
    \begin{itemize}
    \item Скорость выполнения кода
    \item Небольшая экосистема
    \item Не мейнстрим
    \end{itemize}
  \item Недостатки Эликсир
    \begin{itemize}
    \item Проблемы в дизайне языка
    \item Динамическая типизация
    \item Разница между исходным кодом и байткодом
    \end{itemize}
  \end{itemize}
\end{frame}

\begin{frame}
  \frametitle{Скорость выполнения кода}
  Языки по скорости выполнения:
  \begin{itemize}
  \item интерпретируемые (\textbf{Python}, \textbf{Ruby})
  \item компилируемые в байткод (\textbf{Elixir}, \textbf{Java})
  \item компилируемые в нативный код (\textbf{Go}, \textbf{C++})
  \end{itemize}
\end{frame}

\begin{frame}
  \frametitle{Скорость выполнения кода}
  Эликсир быстрее \textbf{Python} и \textbf{Ruby},
  \par \bigskip
  но медленее \textbf{Java} и \textbf{Go}.
\end{frame}

\begin{frame}
  \frametitle{Скорость выполнения кода}
  Это про скорость одного потока на одном ядре.
  \par \bigskip
  А у нас многопоточность.
  \par \bigskip
  Тут важнее средства языка,
  \par \bigskip
  архитектура виртуальной машины и приложения.
\end{frame}

\begin{frame}
  \frametitle{Небольшая экосистема}
  Библиотеки, инструменты и сообщество разработчиков.
\end{frame}

\begin{frame}
  \frametitle{Небольшая экосистема}
  Количество репозиториев на Github:
  \begin{itemize}
  \item Эрланг -- \textbf{20К}
  \item Эликсир -- \textbf{36К}
  \item Ruby -- \textbf{1.5M}
  \item JavaScript -- \textbf{5M}
  \end{itemize}
\end{frame}

\begin{frame}
  \frametitle{Небольшая экосистема}
  Качество важнее количества.
\end{frame}

\begin{frame}
  \frametitle{Не мейнстрим}
  Компании не хотят использовать технологию, в~которой~мало~разработчиков.
  \par \bigskip
  Это риск для бизнеса.
  \par \bigskip
  Разработчики не хотят изучать технологию, которую~используют~мало~компаний.
  \par \bigskip
  Это риск для карьеры.
\end{frame}

\begin{frame}
  \frametitle{Проблемы в дизайне языка}
  Тщательно продуманные дизайн, семантика,
  \par \bigskip
  синтаксис и реализация -- это не про Эликсир.
  \par \bigskip
  (мнение Эрлангистов :)
\end{frame}

\begin{frame}[fragile]
  \frametitle{Проблемы в дизайне языка}
  Вызов функций без круглых скобок:
  \begin{lstlisting}
    my_fun(arg1, arg2)
    my_fun arg1, arg2
  \end{lstlisting}
\end{frame}

\begin{frame}[fragile]
  \frametitle{Проблемы в дизайне языка}
  Два синтаксиса для словарей:
  \begin{lstlisting}
    m1 = %{a: 42}
    m1.a

    m2 = %{"a" => 42}
    m2["a"]
  \end{lstlisting}
\end{frame}

\begin{frame}[fragile]
  \frametitle{Проблемы в дизайне языка}
  Однострочный \textit{do..end}:
  \begin{lstlisting}
    my_fun(a, b), do: a + b

    my_fun(a, b) do
      a + b
    end
  \end{lstlisting}
\end{frame}

\begin{frame}[fragile]
  \frametitle{Проблемы в дизайне языка}
  Вызов анонимной функции:
  \begin{lstlisting}
    my_fun = fn(a) -> a * a end
    my_fun.(42)
  \end{lstlisting}
\end{frame}

\begin{frame}
  \frametitle{Проблемы в дизайне языка}
  Хорошая новость в том, что ко всему привыкаешь :)
\end{frame}

\begin{frame}
  \frametitle{Динамическая типизация}
  Жозе Валим активно работает
  \par
  над внедрением статической типизации.
  \par \bigskip
  \textbf{Dialyzer} -- статический анализ и проверка типов.
  \par
  Ограничен в возможностях, не интегрирован с компилятором.
\end{frame}

\begin{frame}
  \frametitle{Динамическая типизация}
  Языки для BEAM со статической типизацией:
  \par \bigskip
  \textbf{Alpaka}, \textbf{Gleam}, \textbf{Haskerl}, \textbf{Caramel}.
  \par \bigskip
  Не зрелые, не для реальных проектов.
\end{frame}

\begin{frame}
  \frametitle{Разница между исходным кодом и байткодом}
  Абстракции высокого уровня,
  \par \bigskip
  широкое использование макросов.
  \par \bigskip
  Код сильно отличается от байткода.
\end{frame}

\begin{frame}
  \frametitle{Разница между исходным кодом и байткодом}
  Трудно понять, что именно работает в рантайме.
  \par \bigskip
  Это усложняет интроспекцию и трассировку.
\end{frame}

\end{document}
